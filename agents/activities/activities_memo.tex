\documentclass[a4paper,12pt]{report}
\usepackage[spanish]{babel}
\usepackage[utf8]{inputenc}
\usepackage{eurosym}

% Definimos los ifs para el pdf
%\newif\ifpdf
%\ifx\pdfoutput\undefined
%        \pdffalse
%\else
%        \pdfoutput=1
%        \pdftrue
%\fi
%
%%En caso de pdflatex usamos los paquetes con el modificador
%\ifpdf
        \usepackage[pdftex]{graphicx}
        \usepackage[pdftex]{hyperref}
        \pdfcompresslevel=9
%\else
%        \usepackage{graphicx}
%        \usepackage{hyperref}
%\fi

\hypersetup{
pdfauthor={GUL UC3M},
pdftitle={Memoria de actividades del Grupo de Usuarios de Linux de la Universidad Carlos III de Madrid},
pdfsubject={Curso #{CURSO}},
pdfpagemode={None},
pdfview={FitH},
pdfstartview={FitH},
colorlinks={true},
pdfhighlight={/P},
urlcolor={blue},
}

%\author{textbf{secretario@gul.uc3m.es}}
\title{Memoria de actividades #{CURSO}\\ del\\ Grupo de Usuarios de Linux\\ de la\\ Universidad Carlos III de Madrid}
\pagestyle{headings}
\begin{document}
\maketitle
\begin{enumerate}
\item \textsf{\Large Datos relativos a la asociación}
\begin{itemize}
\item{Nombre:} Grupo de usuarios de Linux de la Universidad Carlos III de Madrid
\item{Dirección:} Avda. de la Universidad, 30, edificio Sabatini, despacho 2.3C05
\item{Teléfono:} +34 916249053
\item{Web:} http://www.gul.uc3m.es
\item{FTP:} ftp://ftp.gul.uc3m.es
\item{Correo eletrónico de la junta directiva:} junta@gul.uc3m.es
\item{Número de registro de la Comunidad de Madrid:} 23500
\item{Número de registro del Ayuntamiento de Leganés:} 795
\item{C.I.F.:} G83296988
\item{Número de socios:}
\begin{itemize}
\item{Personas en la lista de correo:} #{ALTAS_EN_LISTA}
\item{Altas en el libro de socios:} #{ALTAS_EN_LIBRO}
\end{itemize}
\item{Composición de la junta directiva:}
\begin{itemize}
\item{Presidente:} #{PRESIDENTE}
\item{Vicepresidente-Secretario:} #{VICEPRESIDENTE}
\item{Coordinador de actividades:} #{COORDINADOR}
\item{Tesorero:} #{TESORERO}
\item{Vocal:} #{VOCAL}
\end{itemize} 

\newpage

\item{Medios materiales:}
\begin{itemize}
<?py if len(INVENTARIO) == 0: ?>
\item No hay datos
<?py else: ?>
<?py     for o in INVENTARIO: ?>
<?py         amount, what = o ?>
\item{} #{amount} #{what}
<?py     #endfor ?>
<?py #endif ?>
\end{itemize}
\end{itemize}
\newpage
\item \textsf{\Large Actividades desarrolladas durante el curso #{CURSO}}
\newline
\newline
\textbf{Es destacable que todas estas actividades han sido realizadas de forma
gratuita y con acceso libre a todos los miembros de la Comunidad Universitaria, así como a personas externas a al universidad.}
\begin{itemize}
\item Cursos del GUL
\begin{description}
\item[Explicación:] El GUL organiza
unos cursos de dos horas cada uno sobre herramientas, lenguajes y temas
sociales relacionados con el software libre. Los cursos impartidos han sido:
<?py for c in CURSOS: ?>
<?py     datemin, datemax, lectures = c ?>
<?py     months = ("enero", "febrero", "marzo", "abril", "mayo", "junio", "julio", "agosto", "septiempre", "octubre", "noviembre", "diciembre") ?>
<?py     month = months[int(datemin.split("-")[1])] ?>
\begin{itemize}
\item Jornadas de #{month}:
\begin{itemize}
<?py     for l in lectures: ?>
\item #{l}
<?py #endfor ?>
\end{itemize}
\end{itemize}
<?py #endfor ?>
Muchos de estos cursos se grabaron y se han puesto en el servidor FTP para que cualquiera pueda verlo.
Desde 2007 toda la organización de las jornadas está reflejada en la web http://cursos.gul.es
\end{description}
\item Servidor FTP
\begin{description}
\item[Explicación:] 
Un servidor FTP es un modo de poner públicamente disponibles contenidos multimedia y software.
Dada la gran interrelación que existe entre el
mundo del sistema operativo Linux e internet nuestro servidor FTP
permite la divulgación de programas y documentación sobre software
libre. Con este espíritu ha estado funcionando el servidor FTP, 
siendo \emph{servidor oficial} de 
Debian GNU/Linux (una de las versiones de Linux más importantes) , teniendo una réplica completa. Durante el año 2010 este servidor fue el servidor oficial de Debian para España. Esta réplica es usada a diario por alumnos, profesores y departamentos.
Este servidor también se utiliza para poner a disposición de todo
el mundo las presentaciones que se utilizan en los cursos organizados
por el GUL, así como el vídeo y/o audio de estos cursos que se han 
podido grabar.
\item[Localización:] http://ftp.gul.uc3m.es
\end{description}
\pagebreak
\item Lista de distribución
\begin{description}
\item[Dirección:] gul@gul.uc3m.es
\item[Explicación:] En la lista de distribución los socios conversan sobre
temas relacionados con la informática y dan solución a las preguntas que
se realizan. Esta lista es la forma primordial por la que los miembros del GUL se relacionan entre sí, a persar del tiempo y la distancia.
Esta lista puede consultarse también por web en la dirección:
http://gul.uc3m.es/pipermail/gul/

En estos estos momentos la lista de correo tiene #{ALTAS_EN_LISTA} miembros de varios continentes.

\end{description}
\item Planeta del GUL
\begin{description}
\item[Explicación:] Con el auge de los diarios electrónicos («blogs»), en
el GUL se ha montado un planeta que auna los «blogs» de todos aquellos
socios que lo pidan. De este modo, se aunan los blogs de varios miembros con inquietudes similares.
\item[Localización:] http://planeta.gul.es
\end{description}
\item Página de documentación
\begin{description}
\item[Explicación:] Se ha iniciado una página de documentación dentro de nuestro
 sitio web, donde se enlaza a documentacion útil para principiantes y documentos
mas extensos sobre más técnicos. Esta página puede ser de gran ayuda para toda aquella gente que no sabe bien como iniciarse en el mundo del software libre.
La página se puede ver en:\\
http://wiki.gul.es/doku.php?id=wiki:documentacion
\end{description}
\item Reuniones
\begin{description}
\item[Explicación:] El segundo viernes de cada mes se organiza una reunión
general del GUL invitando a todo el quiera asistir y así proponer y discutir
proyectos.
\end{description}
\item Grabaciones de las Jornadas Técnicas
\begin{description}
\item[Explicación:] El audio y el vídeo de las charlas de las Jornadas Técnicas han sido montados y tratados para mejorar su calidad, estando a disposición de todo el que quiera verlos u oirlos en el servidor ftp de la asociación, descrito anteriormente.
\end{description}

\pagebreak
\item Programa de radio
\begin{description}
\item[Explicación:] Desde 2010 varios miembros de la asociación han realizado el programa de radio "HolaMundo" de divulgación científica y tecnológica, dentro de la organización ECOLeganés.
A partir de septiembre de 2012 la propia asociación produce autónomamente el programa, emitiéndose por internet mediante los medios disponible en la asociación.
\item[Localización:] http://radio.gul.es
\end{description}

\end{itemize}
\newpage
\item \textsf {\Large Actividades económicas}
\\
\\
\textsf{Ingresos}

\begin{itemize}
<?py if len(INGRESOS) == 0: ?>
\item No hay datos
<?py else: ?>
<?py     for i in INGRESOS: ?>
<?py         date, amount, what = i ?>
\item #{date}: #{amount} Euros (#{what})
<?py     #endfor ?>
<?py #endif ?>
\end{itemize}

\textsf{Gastos}

\begin{itemize}
<?py if len(GASTOS) == 0: ?>
\item No hay datos
<?py else: ?>
<?py     for g in GASTOS: ?>
<?py         date, amount, what = g ?>
\item #{date}: #{amount} Euros (#{what})
<?py     #endfor ?>
<?py #endif ?>
\end{itemize}

\textsf{Saldo actual}

\begin{itemize}
\item #{SALDO} Euros
\end{itemize}


\end{enumerate}
\end{document}
